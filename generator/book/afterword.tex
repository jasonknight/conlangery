\chapter*{OOC Afterword }
\addcontentsline{toc}{chapter}{OOC Afterward }

\begin{quote}
People like to be told what they already know.  Remember that.  They get uncomfortable when you tell them new things.  New things…well, new things aren’t what they expect.  They like to know that, say, a dog will bite a man.  That is what dogs do.  They don’t want to know that a man bites a dog, because the world is not supposed to happen like that.  In short, what people think they want is \textit{news}, but what they really crave is \textit{olds}. - \textbf{Terry Pratchett}
\end{quote}

In designing an alien world, it is important to remember not to make it too alien. This is the mantra of most concept designers. When they imagine alien machines, or buildings, or insects they know that they are constrained by two things, physics and biology, and how their audience \textit{thinks} they understand physics and biology\footnote{Blast Ended Skrewts notwithstanding}. You are always limited by what your audience can believe is possible. It's much broader than you think, but it is not infinite. The suspension of disbelief can never be pushed so far that it becomes the suspension of reason.

The limits on what is conceivable for a culture would seem to not have this issue. Why should an alien culture adhere to any human sociological, psychological, or even ethical standards? You would think that you have a much wider field in which to maneuver, but you would in fact be completely wrong. The purpose of a successful alien culture, that is one where people would want to dress up like them, learn their con-lang, and fantasize about adventures on their homeworld, is not to be alien, but to be a mirror. We want attributes, mores, and norms that we find inside ourselves magnified and reflected back to us. No one wants to visit the Borg homeworld. Unless they have a cyborg-amputee fetish. The Borg are too alien.\footnote{Until the introduction of the Borg Queen, the Borg were very nebulous baddies.} I imagine few people fantasize about romantic encounters in the assimilation chamber. Voyager and later TNG movies went a long way to humanizing the Borg. Alice Krige in the role of The Borg Queen is a case in point. The hive-mind of the Borg became a little more litteral, and suddenly people were much more interested in the social structure of the Borg. They still haven't been completely rehabilitated, but I imagine future Star Trek will retcon them into pointlessness.

\clearpage

\begin{quote}
I must tell you I'm disappointed at hearing you mouth the usual platitudes of peace and friendship regarding an implacable foe like the Romulans. But I live in hope that you may one day see the universe for what it truly is, rather than what you'd wish it to be. - \textbf{Elim Garak}
\end{quote}

\begin{quote}
Well, I shall endeavor to become more cynical with each passing day, look gift hordes squarely in the mouth, and find clouds in every silver lining. - \textbf{Julian Bashir}
\end{quote}

\begin{quote}
If only you meant it. - \textbf{Elim Garak}
\end{quote}

Even in the case of the Vulcans, the Federations closest Allies, they are constantly presented as less than humans, with much to learn about the right way of existing. Vulcans have been heavily retconned into incredulity. Despite their outward appearance of dispassion and logic, they are consistently portrayed as highly emotional, or in desperate need of some kind of emotional intelligence that can usually only be provided by a human counterpart. It's rather sad that their entire culture is based on such a false premise as the supremacy of logic and dispassionate scientific enquery. Enterprise made immense strides in sexualising the Vulcans as well, and the character T'Pol is often bordering on ridiculous with her constant posturing in cat suits. 

Enterprise didn't stop there in its effort to degrade the Vulcan Culture where they are presented as stale, self-serving, and propagandist. Apparently there are even emotionally \textit{free} Vulcans. T'Pol needs to be instructed on how to perform mind melds by a human. Apparently the thin veneer of logic can even be broken by jazz music! It's all a bit ridiculous. Considering the lack luster support for Star Trek, I can only conclude that this trend is having a predominantly negative effect on the franchise. The latest Star Trek movies, while box office smashes, are beneath consideration in every possible way.

It is essential that an alien race only diverge at a few key points, and their divergence must be counter-balanced by the enlargement of at least one positive feature. The \klingons may be violent, loud, and filthy but their oversized sense of honor and loyalty make them more palatable. While Kirk makes the comparison between the \klingons and Nazi Germans in \movie{Undiscovered Country}, they really aren't that way at all, and never could be. If there is not a key feature enlarged to compensate, then we must at least observe a motivation to change the negative aspects of the culture. The Ferengi would be intolerable if it weren't for Moogie and the progressive Zek. The fact that women are, by human standards, horribly objectified, sexualized, and enslaved would be completely intolerable if there wasn't some feminist effort to change. It does leave one to wonder, would an alien races female members ever be happy in servitude? Of course not, because that is one of the many taboos or morals of our culture. It is very important that no serious taboo or too widely accepted moral be broken or assailed unless there is a strong counter-balancing element. 

A taboo is a prohibition based on the belief that the behavior is either too sacred or too accursed for people to engage in it. While human culture doesn't have any universal taboos, there are a number of them that are particularly strong in the west. The english speaking west being the most likely consumer of Star Trek related material.

\begin{enumerate}
	\item Incest
	\item Necrophilia
	\item Adultery
	\item Fornication
	\item Pedophilia
	\item Beastiality
	\item Cannibalism
	\item Infanticide
	\item Murder
\end{enumerate}

It is not that people don't commit taboos, it is that they will 1) feel uncomfortable discussing them 2) conceal them or any knowledge of them, and 3) require an \textit{abjure} statement about them. You know you are talking about a taboo when you feel the need to prefix what you say with, of course \textit{x} is horribly wrong, inexcusable, repugnant, disgusting, and any number of negative adjectives. The more vitriol the better.

Different classes of people may also have taboos against inter-racial marriage, homosexuality and even religious inter-marriage. Humans have a suspiciously large number of sexual and mating taboos.

Another important feature of a society is its morals, in the sense of things you have learned, or think you have learned. These are tacit truths about human behavior. You know you are dealing with a moral when you find yourself prefixing statements with \textit{Obviously ...} and \textit{Everyone knows/accepts/understands that}.

\begin{enumerate}
	\item It is better to be poor than rich, or at least the rich must be very unhappy
	\item Violence is never the answer. Those who live by the sword, die by the sword
	\item Everyone makes mistakes, even the government, that's why democracy is the best system \ldots
\end{enumerate}

The Ferengi obviously violate number one, being obscessed with money, they are unhappy, and Quark always finds himself most satisfied when he violates Ferengi cultural mores and increases wages, or grants workers rights, or makes some sort of donation to the Bajoran Orphan's Fund. The \klingons violate number two, and usually end up being reasoned to a more peaceful solution. One begins to wonder how they've survived this long without the Federation there to explain that one too them. The Cardassians violate number three, and a great deal of plot fodder centers around their totalitarian political system. They are just about to go democratic when the Dominion sweeps in. We are left with the impression from Garak at the end of DS9 that things will be changing for the better. Other ones violated are those like 1) women are intrinsically equal to men(the Ferengi), 2) People are innocent until proven guilty(Cardassians), and 3) pride goeth before a fall(Romulans).

Morals are sometimes a bit timeless, that is they have been around for a long time, and others are era dependent. Everyone knows that smoking is bad, obviously eating too much fat can cause a heart attack and so on. Morals are not right or wrong by virtue of being a moral. Sometimes everyone knows what is right to be right and sometimes everyone knows what is wrong to be right. Critical thinking is a must.

\begin{quote}
No one can be a great thinker who does not recognise, that as a thinker it is his first duty to follow his intellect to whatever conclusions it may lead. Truth gains more even by the errors of one who, with due study and preparation, thinks for himself, than by the true opinions of those who only hold them because they do not suffer themselves to think. - \textbf{John Stuart Mill}
\end{quote}

With the exception of Farscape and the Sebaceans, Humans are generally the good guys, and everyone else must invariably be less than us. No one wants to fantasize about their race being seen as a pariah on the galaxy after all. Humans must invariably triumph, and most importantly they must triumph because of what we imagine are our good qualities. We must win because we are optimistic, persevering, benevolent and peace loving\footnote{Opinions that human history sutbbornly refuses to bare out.}. It is very important that we protect ourselves from our inherent weaknesses and moral and ethical frailties. Even though DS9 made great strides to undo fascist starfleet propaganda by mixing in some conspiracies, and even an ultra-secret spy death squad Section 31\footnote{Even in this case, all the build up was undone in \textbf{Inter Arma Enim Silent Legis}}, these are always to be presented as the exception rather than the rule. Any evil we do must be portrayed as an honest mistake, a well intentioned genocide here, and unfortunate world war there. But let's not descend too far into human hating. It is not that humans are all bad or all good, but more complex than we are willing to admit. So must be other races.

What is great about humanity is that there is something for everyone, even evil people. Of course, Hitler didn't wake up in the morning, look in the mirror and say \"Today I am going to be evil!\". He probably thought he was a great guy. Most of history doesn't agree, but that's more common than you think. Obviously there are people in this world who like taking candy from babies, metaphorically and perhaps literally as well. The possibilities of naughtiness abound. From kink to killing, somewhere and somewhen there is a place for every type of human being. We may not like it, but there it is. It is what makes humanity something worth existing. A utopia would suck for most people, as Pat Califia noted\footnote{I am here referring to her essay Whoring in Utopia.}.

As it stands, the world of today would invariably be the \#1 tourist destination for the entire galaxy, because for a price almost anything can be had. In the end, what makes a society interesting and immersive is that it is some place you'd want to visit. Most alien societies fail this test in spades. No person in their right mind would ever want to visit Kronos as it is described, or Romulus, or Ferenginar. Vulcan wouldn't be much of a tourist stop either come to think about it. Unless you are really into dry formalized debates and lots of meditating, but then that's only a small portion of the entire human population. The question you should ask about your alien society is: How likely is MTV to want to film a spring break special there\footnote{I am only slightly kidding.}?

This all depends on what kind of society you want to create. You may just need a bunch of paper tiger baddies. Maybe your intention is to create a society for inverse comparison against humans? Maybe you don't really care all that much. It's like a house. In a pinch just about any structure will do against the rain, but would you actually want to live there? 





