\chapter*{Grammar }
\addcontentsline{toc}{chapter}{Grammar }

\section*{Nouns}
\addcontentsline{toc}{section}{Nouns, Pronouns and Adjectives}

\klingonl groups nouns and pronouns together. There are only single pronouns, Me, You, Him, Her, It, Them, That.\par


\begin{table}[ht]
	\caption{Personal Pronouns}
	\centering
	\begin{tabular}{ l l }
		\toprule
		\english				& 	\klingonl 		\\
		\midrule
		I/Me/Myself 			&	u 				\\
		You(singular) 			&	g\'{a} 			\\
		You(plural) 			&	ng\'{a} 		\\
		He/Him 					&	jot 			\\
		She/Her 				&	sa'an 			\\
		it 						&	kan 			\\
		it(indefinite) 			&	la 				\\
		Them/They 				&	ku'ul 			\\
		This 					&	suj 			\\
		These 					&	sujm\'{a} 		\\
		That 					&	soj 			\\
		Those 					&	sojm\'{a} 		\\
		Here 					&	elov 			\\
		There 					&	ya 				\\
		\bottomrule
	\end{tabular}
	\label{table:personal_pronouns}
\end{table}

Nouns agglutinate certain particles or even other nouns and pronouns that can modify them. In the case of pronouns. Consider this table.\par

\begin{table}[ht]
	\caption{Impersonal Pronouns}
	\centering
	\begin{tabular}{ l l l l   }
		\toprule
				& Interrogative 			& Demonstratives 			& 				\\	
		\midrule
		Adj. 	& which 					& this 			 			& that 			\\	
				& suja'na/soja'na 			& suj 						& soj 			\\	
		Person 	& who 						& this(one) 				& that(one) 	\\
				& klina					 	& klin suj 					& klin soj		\\
		Thing 	& what 						& this(one) 				& that(one) 	\\
				& kana		     			& suj 						& soj 	 		\\
		Place 	& where 					& here 						& there 		\\
				& ya'na 					& elov		 				& ya 			\\
		\bottomrule
	\end{tabular}
	\label{table:impersonal_pronouns_1}
\end{table}

When referring to unknown or uncertain quantities, like some, \vocab{la} is used. \vocab{la} is the 
indefinite actor used also in the passive voice \phrase{ng\'{a} la zhal - you are remembered}.

\begin{table}[ht]
	\caption{Impersonal Pronouns Cont.}
	\centering
	\begin{tabular}{ l l l l }
		\toprule
				& Indefinite 			& 					& 							\\
		\midrule
		Adj. 	& some 					& none 				& every 					\\
				& la 		  			& la'b\'{a} 		& sor 						\\
		Person 	& someone 				& no one 			& everyone 					\\
				& laklin 				& lab\'{a}klin 		& sora'klin 				\\
		Thing 	& something 			& nothing 			& everything 				\\
				& lakan			 		& lakanb\'{a} 		& sorkan 					\\
		Place 	& somewhere 			& nowhere 			& everywhere 				\\
				& laya 					& layab\'{a} 		& sorya 					\\
		\bottomrule
	\end{tabular}
	\label{table:impersonal_pronouns_2}
\end{table}

As we see in Table \ref{table:impersonal_pronouns_1} and Table \ref{table:impersonal_pronouns_2} the pronouns glue together to into a compound pronoun like \defn{sor}{all} with \defn{ya}{there} to form \defn{sorya}{everywhere}. In the case of \defn{klina}{who} we have an example of the disappearing \'n\' in \klingonl as this is a compound pronoun with \defn{klin}{a particle having to do with people} and \defn{na}{interrogative particle}. Notice the exception in Table \ref{table:impersonal_pronouns_1} where the compound pronoun does not agglutinate but is isolating and inverted. You would expect \vocab{sujklin} or even \vocab{klinsuj} here. This is one of the many exceptions to be found in \klingonl, due to the hodgepodge nature of \klingonl source languages. \place{Sig\'{o}r} province is known as the Savoyan Source \vocab{thadon \name{savoy}}, and it's features are agglutination and OSV word order. \place{Mithrot} province provides the Mithrotian Source \vocab{thadon \place{mithrot}} which has distinctly Latin features in that verbs are inflected, but only for person and number. 

\klingonl was all but destroyed during the \name{hurk} invasion in 940 C.E.\footnote{\english dating is used in this text for the readers convience.}. Once the invaders had been repelled\footnote{The \name{hurk} empire crumbled from within, it would be several millenia before \klingons would admit that they had little to do with getting rid of them.}, \klingons wanted to recover their lost heritage, but the \name{hurk} suppression of traditional culture and language had been so complete, and the \klingons were in such a state of economic and cultural ruin, that they hardly knew where to start. Almost all literary works had been burnt, inscriptions scraped out, temples demolished. What did survive was extremely old and practically useless. \klingons were mixed ethnically, in an attempt to insure that no nonds of culture could be maintained, and what passed for \klingonl at that time was little more than a pidgin dialect.

The wise elders of the day did the only thing they could. They made it all up. They had a wide variation of pidgin dialects, very few examples of old \klingon and one or two inscriptions and so they went about recreating the entire language from scratch.  When this was complete a dictionary was made and any missing words were added on the fly. Because a huge quantity of words in the lexicon were \name{hurk}, they had to be removed and new \klingon words were generated. Some more consistently and well thought out than others.

The \name{hurk} occupation of \name{kronos} had scarred and mangled the \klingon spirit, but they were resolved to be reborn from the ashes with a philosophy of never again being crushed under the boot of an invader\footnote{Unfortunately this didn't extend to crushing each other under \klingon boots. Shortly after throwing off the \name{hurk}, \klingons entered the Five Tyrants Period.}. In this new spirit of comraderie and equality it was decided that everyones linguistic peculiarities were at least partially right, and so a great many exceptions and deviations made it into the language to accommodate the modes of speech of various regions.

\begin{table}[ht]
	\caption{Noun Particles and Suffixes}
	\centering
	\begin{tabular}{ l l l }
		\toprule
		a- & N-yu & predictive progressive, afloat, awash \\
		ambi- & N-nu & both \\
		amphi- & otor N & around, surround \\
		ana- & N-ede & up, back, anew \\
		ano- & N-ede & up, back, anew \\
		ante- & krim N & before \\
		ant- & krim N & before \\
		anti- & N-sus & against,opposite \\
		apo- & suru N & away from \\
		arch- & N-'os & supreme, highest, best, biggest \\
		be- & N-vav & equipped with, covered with \\
		bi- & N-nu & both of two \\
		bibl- & vral N & book, having to do with books \\
		bibli- & vral N & book, having to do with books \\
		biblio- & vral N & book, having to do with books \\
		bio- & N-oke & having to do with biological things \\
		cad- & N-jon & to seize \\
		cap- & N-jon & to seize \\
		caus- & N-mek & burn, heat \\
		circum- & otor N & around, about, surrounding \\
		co- & utsul N & joint, accompanying, with \\
		cog- & utsul N & joint, accompanying, with \\
		col- & utsul N & joint, accompanying, with \\
		com- & utsul N & joint, accompanying, with \\
		con- & utsul N & joint, accompanying, with \\
		contra- & N-sus & against,opposite \\
		counter- & N-sus & against,opposite \\
		corp- & torgh N & body \\
		de- & N-lus & not, reverse action, opposite \\
		dec- & N-mah & ten, 10 \\
		deca- & N-mah & ten, 10 \\
		dis- & N-lus & not, opposite of \\
		dif- & N-lus & not, opposite of \\
		dict- & jatl N & speaking, saying \\
		dei- & N-kun & god \\
		divi- & N-kun & god \\
		div- & N-lis & separate \\
		demo- & klin N & people \\
		doc- & N-goj & teach \\
		dual- & N-nu & both of two \\
		dys- & luja N & fail, fault \\
		ec- & N-dek & out of, from \\
		eco- & hyul N & household, environment, local area \\
		ecto- & N-dek & out of, from \\
		en- & N-bal & get into, enmesh \\
		endo- & N-æs & within, inside \\
		em- & N-bal & get into, put into, empower \\
		ex- & N-mu & former \\
		\bottomrule
	\end{tabular}
	\label{table:prefixes_1}
\end{table}

As we see with Table \ref{table:prefixes_1} \& \ref{table:prefixes_2} there is a correlation between \english prefixes and \klingon particles. This is a general table of rough equivalents, there really isn't a 1-to-1 correlation, and many compound \english words that take a prefix or suffix have a their own word in \klingon that doesn't include particles from these tables.

\begin{table}[ht]
	\caption{Noun Particles and Suffixes Cont.}
	\centering
	\begin{tabular}{ l l l }
		\toprule
		lang- & áse N & having to do with language, a tool \\
		fore- & krim N & before, in front \\
		hind- & adah N & after \\
		mal- & usul N & bad, badly \\
		mid- & hraj N & middle, between \\
		mini- & N-om & small, miniature \\
		mis- & N-'œ & wrong, astray \\
		out- & N-'os & better, faster \\
		over- & hotsil N & excessive, above \\
		peri- & ulari N & around, near, nearby \\
		post- & N-ada & after,behind \\
		pro- & N-guj & for, on the side of \\
		pre- & krim N & before, in front \\
		pseudo- & N-som & fake, false, misrepresent \\
		re- & N-gul & again \\
		self- & N-ag & self, self-sufficient \\
		sep- & N-lis & divied, pull apart \\
		soci- & N-dat & having to do with society \\
		sub- & N-om & below \\
		sup- & N-om & below, under \\
		super- & N-'os & above, higher \\
		supra- & N-'os & above, higher \\
		sur- & N-'os & above, higher \\
		ultra- & N-'os & above, higher \\
		tele- & N-kuk & distance, range, from a distance \\
		trans- & rats N & through, across \\
		un- & N-sus & not, against \\
		uni- & N-mev & one, together, united \\
		under- & N-om & lower, lesser, under, beneath \\
		up- & N-'os & greater, higher \\
		with- & N-sus & against, withstand \\
		\bottomrule
	\end{tabular}
	\label{table:prefixes_2}
\end{table}
As we see in Table \ref{table:suffixes_1} there are also rough equivalents to \english suffixes. 
\begin{table}[ht]
	\caption{Noun Particles and Suffixes Cont.}
	\centering
	\begin{tabular}{ l l l }
		\toprule
		-an & N-yi' & one who does \\
		-ance & N-jub & action of state of \\
		-ancy & jub' N & state, quality, or capacity \\
		-ant & N-si' & something that does, performs \\
		-ard & dokjo N & characterized as, like \\
		-ary & dokjo N & resemble, like \\
		-cide & hotoh N & kill \\
		-ent & N-si' & something that does, performs \\
		-fold & dokjir N & in a manner marked by, fourfold \\
		-ful & N-boh & filled, full with \\
		-fy & bakha N & make, form \\
		-ish & dokjo N & resemble, like \\
		-ism & N-kul & doctrine, conduct, belief \\
		-ist & N-yi' & one who does \\
		-ology & N-ked & the science of \\
		-nomy & okedi N & the science of naming, having to do with names \\
		-onymy & ási N & specified, named, or naming \\
		-onym & ási N & specified, named, or naming \\
		-language & áse N & tool, language \\

		\bottomrule
	\end{tabular}
	\label{table:suffixes_1}
\end{table}



\clearpage

\section*{Verbs }
\addcontentsline{toc}{section}{Verbs and Adverbs }

\klingonl verbs inflect to indicate person, and adverbs which modify them directly follow the verb and agree in person. \defn{sal}{to run} becomes \defn{sesa}{I run, am running}. To say I am running fast you would use the adverb \defn{tsefut}{fast, quickly} as \phrase{sesa tsefesa}. Tense is not inflected in the verb, but it presents as a particle that \textit{usually} appears at the beginning of a phrase as in \phrase{yaisg sesa - I ran} and could be used in a sentence such as \phrase{yaisg porahndak sesa - I ran to the store}. \klingonl tenses are very precise in the short term for future and past, \phrase{soto porahndak sesa - earlier today I ran to the store}. There is nothing that actually requires this word order, but it is the most common. You could say \phrase{otor porahn sesa tsefesa soto - earlier today I ran around the store}. The same would work for \phrase{sesa tsefesa soto otor porahn}. It is considered a sign of poor education to do this. The general structure of \klingonl phrases are Tense, Object, Subject, Verb. Any other order would be a little weird and out of the ordinary, but \klingons have learned to deal with foreigners, and they have a very liberal linguistic heritage. There is no \klingon concept of a \textbf{grammar nazi}.

That being said, there is a right way and a wrong way, limited poetic constructions notwithstanding, you should adhere to the rules as closely as possible. Deviate from them only occasionally, perhaps to emphasize a point.

In \english there is the concept of a phrasal adverb, such as \phrase{I ran to the store quickly.} \klingonl does not do this, it is a strict rule in \klingonl that an adverb follows a verb. The above can be accomplished with a \textit{tense adjective}. Adjectives usually follow nouns, but when they follow a tense word they modify the entire phrase. Consider: \phrase{yaisg tsebas porahndak sesa - I ran to the store quickly}.

\section*{Particles}
\addcontentsline{toc}{section}{Particles}

Up to this point you have seen many particles and suffixes (they are considered in the same class) in action. \klingonl particles are very versatile, they can take as their "argument" any other \klingonl word (including other particles!) and they will consistently modify it. Some combinations while grammatically correct are of course non-sensical. There are som restrictions to this, for instance free floating particles, e.g. those that do not agglut cannot modify an agglutinating particle. Setting two free floating particles next to each other will make the first modify the second, and the combined meaning of those will then modify whatever follows. In some cases, this is not desired, so it is possible to separate the particles with a \defn{pa}{nul particle}.